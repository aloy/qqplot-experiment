\documentclass{article}
\usepackage[margin=1in]{geometry}
\usepackage[colorlinks=true, citecolor=blue, linkcolor=blue]{hyperref}
\usepackage[parfill]{parskip}

\begin{document}

\textbf{Aldor-Noiman, S., Brown, L. D., Buja, A., Rolke, W., \& Stine, R. A. (2013). The Power to See: A New Graphical Test of Normality. \emph{The American Statistician}, 67(4), 249--260. \url{DOI:10.1080/00031305.2013.847865}}

An apt summary of this paper is given in the following snippet from the introduction:
\begin{quote}
This article introduces a simple method that provides simultaneous confidence bands for a normal quantile-quantile (Q-Q) plot. These bands define a test of normality and are narrower in the tails than those associated with the Kolmogorov-Smirnov test. Correspondingly, this new procedure has greater power to detect deviations from normality in the tails.
\end{quote}

Additionally, in Section~3.2, the authors investigate the impact of different parameter estimates on the power of their test (the TS test). The following is a snippet from their findings:

\begin{quote}
These results indicate that for the TS test, one should use the median and $Q_n$ to estimate the location and scale parameters. In general, the TS test with this choice of estimators performs at least as well as the Shapiro--Wilk test (and often much better).
\end{quote}

Here, $Q_n$ denotes the adjusted MAD discussed by Croux and Rousseeuw (1993).

\bigskip


{\bf Hazelton, M. L. (2003). A Graphical Tool for Assessing Normality. \emph{The American Statistician}, 57(4), 285--288. \url{doi:10.1198/0003130032341}}

This paper doesn't really discuss Q-Q plots, rather it discusses the use of a log density plot for distributional assessment.

\bigskip

{\bf Rosenkrantz, W. A. (2000). Confidence Bands for Quantile Functions: A Parametric and Graphic Alternative for Testing Goodness of Fit. \emph{The American Statistician}, 54(3), 185--190. \url{DOI:10.1080/00031305.2000.10474543}}

This snippet from the abstract is a great summary of the article:

\begin{quote}
This article derives simultaneous $100( 1 - \alpha )\%$ confidence intervals for the quantiles of a normal distribution using a method first proposed by Cheng and Iles and independently rediscovered by Satten. These methods yield a novel parametric/graphic alternative to the usual goodness-of-fit tests for normality based on normal Q-Q plots, or extensions of Kolmogorov's test based on the empirical distribution function: Draw the graph of the empirical quantile plot, then plot the corresponding upper and lower bounds for these quantiles. We thus obtain a $100( 1 - \alpha )\%$ confidence band for the empirical quantile plot. The more the parent distribution of the data departs from the normal, the more likely the empirical quantile plot will have points that lie outside these bounds. The empirical quantile plot and its confidence bands are pleasing to the eye and easy to interpret, even for a user unfamiliar with the underlying theory.
\end{quote}

This article also reminded me that the discussion of pointwise confidence bands for Q-Q plots was included in Davison \& Hinkley (1997).

\bigskip

{\bf Gan, F. F., Koehler, K. J., \& Thompson, J. C. (1991). Probability plots and distribution curves for assessing the fit of probability models. \emph{The American Statistician}, 45(1), 14--21.}

This paper gives an overview of the use of P-P plots to assess the fit of probability models. Aside from the development of P-P plots, it points out that ``discrepancies in tail regions may be more pronounced on a Q-Q plot, but a P-P plot may be more sensitive to discrepancies in the `middle' of the hypothesized distribution, such as shifts in a modal region or misspecification of the number of modes.''


\end{document}