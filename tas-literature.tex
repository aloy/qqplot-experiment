\documentclass{article}
\usepackage[margin=1in]{geometry}
\usepackage[colorlinks=true, citecolor=blue, linkcolor=blue]{hyperref}
\usepackage[parfill]{parskip}

\begin{document}

\textbf{Aldor-Noiman, S., Brown, L. D., Buja, A., Rolke, W., \& Stine, R. A. (2013). The Power to See: A New Graphical Test of Normality. \emph{The American Statistician}, 67(4), 249--260. \url{DOI:10.1080/00031305.2013.847865}}

An apt summary of this paper is given in the following snippet from the introduction:
\begin{quote}
This article introduces a simple method that provides simultaneous confidence bands for a normal quantile-quantile (Q-Q) plot. These bands define a test of normality and are narrower in the tails than those associated with the Kolmogorov-Smirnov test. Correspondingly, this new procedure has greater power to detect deviations from normality in the tails.
\end{quote}

Additionally, in Section~3.2, the authors investigate the impact of different parameter estimates on the power of their test (the TS test). The following is a snippet from their findings:

\begin{quote}
These results indicate that for the TS test, one should use the median and $Q_n$ to estimate the location and scale parameters. In general, the TS test with this choice of estimators performs at least as well as the Shapiro--Wilk test (and often much better).
\end{quote}

Here, $Q_n$ denotes the adjusted MAD discussed by Croux and Rousseeuw (1993).

\bigskip


{\bf Hazelton, M. L. (2003). A Graphical Tool for Assessing Normality. \emph{The American Statistician}, 57(4), 285--288. \url{doi:10.1198/0003130032341}}

{\bf Rosenkrantz, W. A. (2000). Confidence Bands for Quantile Functions: A Parametric and Graphic Alternative for Testing Goodness of Fit. \emph{The American Statistician}, 54(3), 185--190. \url{DOI:10.1080/00031305.2000.10474543}}

{\bf Gan, F. F., Koehler, K. J., \& Thompson, J. C. (1991). Probability plots and distribution curves for assessing the fit of probability models. \emph{The American Statistician}, 45(1), 14--21.}



\end{document}